%
%  RULES OF THE GAME
%
%  * 80 characters
%  * line breaks at the ends of sentences
%  * eqnarrys ONLY
%  * that is all.
%

\documentclass[12pt,preprint]{aastex}

\pdfoutput=1

\include{vc}

\usepackage{color,hyperref}
\definecolor{linkcolor}{rgb}{0,0,0.5}
\hypersetup{colorlinks=true,linkcolor=linkcolor,citecolor=linkcolor,
            filecolor=linkcolor,urlcolor=linkcolor}
\usepackage{url}
\usepackage{amssymb,amsmath}
\usepackage{subfigure}

\newcommand{\project}[1]{\emph{#1}}
\newcommand{\kepler}{\project{Kepler}}
\newcommand{\KT}{\project{K2}}
\newcommand{\terra}{\project{TERRA}}
\newcommand{\license}{MIT License}

\newcommand{\paper}{\textsl{Article}}

\newcommand{\foreign}[1]{\emph{#1}}
\newcommand{\etal}{\foreign{et\,al.}}
\newcommand{\etc}{\foreign{etc.}}
\newcommand{\True}{\foreign{True}}
\newcommand{\Truth}{\foreign{Truth}}

\newcommand{\figref}[1]{\ref{fig:#1}}
\newcommand{\Fig}[1]{Figure~\figref{#1}}
\newcommand{\fig}[1]{\Fig{#1}}
\newcommand{\figlabel}[1]{\label{fig:#1}}
\newcommand{\Tab}[1]{Table~\ref{tab:#1}}
\newcommand{\tab}[1]{\Tab{#1}}
\newcommand{\tablabel}[1]{\label{tab:#1}}
\newcommand{\Eq}[1]{Equation~(\ref{eq:#1})}
\newcommand{\eq}[1]{\Eq{#1}}
\newcommand{\eqalt}[1]{Equation~\ref{eq:#1}}
\newcommand{\eqlabel}[1]{\label{eq:#1}}
\newcommand{\Sect}[1]{Section~\ref{sect:#1}}
\newcommand{\sect}[1]{\Sect{#1}}
\newcommand{\sectalt}[1]{\ref{sect:#1}}
\newcommand{\App}[1]{Appendix~\ref{sect:#1}}
\newcommand{\app}[1]{\App{#1}}
\newcommand{\sectlabel}[1]{\label{sect:#1}}

\newcommand{\dd}{\ensuremath{\,\mathrm{d}}}
\newcommand{\bvec}[1]{\ensuremath{\boldsymbol{#1}}}
\newcommand{\appropto}{\mathrel{\vcenter{
  \offinterlineskip\halign{\hfil$##$\cr
    \propto\cr\noalign{\kern2pt}\sim\cr\noalign{\kern-2pt}}}}}
\newcommand{\densityunit}{{\ensuremath{\mathrm{nat}^{-2}}}}

% TO DOS
\newcommand{\todo}[3]{{\color{#2} \emph{#1} TODO: #3}}
\newcommand{\dfmtodo}[1]{\todo{DFM}{red}{#1}}
\newcommand{\hoggtodo}[1]{\todo{HOGG}{blue}{#1}}

\begin{document}

\title{%
    Transiting exoplanet search in the \KT\ light curves
}

\newcommand{\nyu}{2}
\newcommand{\mpia}{3}
\newcommand{\cds}{4}
\newcommand{\mpis}{5}
\author{%
    Daniel~Foreman-Mackey\altaffilmark{1,\nyu},
    David~W.~Hogg\altaffilmark{\nyu,\mpia,\cds},
    Bernhard~Sch\"olkopf\altaffilmark{\mpis},
    Dun~Wang\altaffilmark{\nyu},
    \etal
}
\altaffiltext{1}         {To whom correspondence should be addressed:
                          \url{danfm@nyu.edu}}
\altaffiltext{\nyu}      {Center for Cosmology and Particle Physics,
                          Department of Physics, New York University,
                          4 Washington Place, New York, NY, 10003, USA}
\altaffiltext{\mpia}     {Max-Planck-Institut f\"ur Astronomie,
                          K\"onigstuhl 17, D-69117 Heidelberg, Germany}
\altaffiltext{\cds}      {Center for Data Science,
                          New York University,
                          4 Washington Place, New York, NY, 10003, USA}
\altaffiltext{\mpis}     {Max Planck Institute for Intelligent Systems
                          Spemannstrasse 38, 72076 T\"ubingen, Germany}

\begin{abstract}

The photometry from the \KT\ extension of NASA's \kepler\ mission is wrought
with systematic effects due to the relative imprecision of the telescope
pointing.
We present a method for searching these light curves by simultaneously
fitting for these systematics and the transit signals of interest.
This method is more computationally expensive than standard search algorithms
but we demonstrate that it can be efficiently implemented and used to
discover transit signals in the existing dataset.
We apply this method to the full Campaign 1 dataset and report a list of XXX
planet candidates transiting YYY stars.

\end{abstract}

\keywords{%
methods: data analysis
---
methods: statistical
---
catalogs
---
planetary systems
---
stars: statistics
}

\section{Introduction}

\section{Photometry}

The starting point for analysis is the raw pixel data.
We download the full set of XXX target pixel files for \KT's Campaign 1 from
MAST\footnote{\url{https://archive.stsci.edu/k2/}}.
Then, we extract photometry using fixed circular apertures of varying sizes
centered on the mean position of the brightest star (on average) in each
frame.
The centroids and initial flux estimates were measured using
\project{simplexy}, a component of the \project{Astrometry.net} pipeline
\citep{astrometry}.
For each time series, we used 20 circular apertures ranging in radius from 0.5
to 10 pixels and---following \citet{vanderberg-a}---chose the aperture size
that resulted in the smallest CDPP \dfmtodo{CITE CDPP} with a 6 hour
window.\footnote{Note that although we chose a specific aperture, photometry
for every aperture size is available online at \dfmtodo{some URL}.}

Most methods for analyzing \KT\ data---and \kepler\ data, for that
matter---involves some sort of pre-processing or ``de-trending'' step.
For example, \citet{vanderberg-a} measure the centroid motion for each star
and regress out any signal in their aperture photometry that can be fit by
this motion.
Similarly, \citet{crossfield} iteratively construct a robust Gaussian Process
model for the photometry as a function of the measured centroids and de-trend
using the mean prediction from that model.
In our analysis, we don't do any further preprocessing of the light curves
because, as we describe in the next section, we simultaneously fit for the
trends and the transit signals that we are searching for.

One key realization that is also exploited by the official \kepler\ pipeline
is that the systematic trends due to pointing shifts and other instrumental
effects are shared---with different signs and weights---by all the stars on
the detector.
To exploit this fact, the \project{PDC} component of the \kepler\ pipeline
removes any trends from the light curves that can be fit using a linear
combination of a small number of ``eigen-light curves'' (ELCs) found by
running Principal Component Analysis (PCA) on a set of light curves
\citep{map-pdc1, map-pdc2}.
Similarly, we ran PCA (as implemented by the \project{scikit-learn} project
\citealt{sklearn}) on the full set of Campaign 1 light curves to determine a
basis of representative ELCs.
We chose to keep the top 200 components---many more than are normally
used---for our model described in the following section.
\dfmtodo{do we want to plot some of these light curves here?}


\section{Joint transit \& variability model}

The key insight in our transit search method that sets it apart from the
other standard procedures is that no de-trending is necessary.
Instead, we can fit for the noise (or trends) and signal simultaneously.
This is theoretically appealing because it will be more sensitive to low
signal-to-noise transits.
The main reason for this is that the signal is not exactly orthogonal to the
systematics and the de-trending will over-fit causing the signal to be
distorted and weaker.
In order to reduce this effect, most procedures use a very rigid model for
the trends.
For \KT, this has been implemented by asserting that the centroids contain
all of the information needed to describe the trends.
In the \kepler\ pipeline, this is implemented by only allowing a small number
of PCA components to contribute to the fit.

To avoid over-fitting in our pipeline, we instead simultaneously fit for the
transit signal and the trends using a rigid model for the signal and a
relatively flexible model for the noise.



\section{Search pipeline}

\section{Results}

\section{Discussion}

\acknowledgments
It is a pleasure to thank
\ldots\
for helpful contributions to the ideas and code presented here.
This project was partially supported by the NSF (grant AST-0908357), NASA
(grant NNX08AJ48G), and the Moore--Sloan Data Science Environment at NYU.
This research made use of the NASA \project{Astrophysics Data System}.

\newcommand{\arxiv}[1]{\href{http://arxiv.org/abs/#1}{arXiv:#1}}
\begin{thebibliography}{}\raggedright

\bibitem[Crossfield \etal(2015)]{crossfield}
Crossfield, I.~J.~M., Petigura, E., Schlieder, J., \etal\ 2015,
\arxiv{1501.03798}

\bibitem[Lang \etal(2010)]{astrometry}
Lang, D., Hogg, D.~W., Mierle, K., Blanton, M., \& Roweis, S.\ 2010, \aj, 139,
1782

\bibitem[Pedregosa \etal(2011)]{sklearn}
Pedregosa, F., Varoquaux, G., Gramfort, A., \etal\ 2011, JMLR, 12, 2825

\bibitem[Smith \etal(2012)]{map-pdc2}
Smith,~J.~C., Stumpe,~M.~C., Van Cleve,~J.~E., \etal\ 2012,
\pasp, 124, 1000

\bibitem[Stumpe \etal(2012)]{map-pdc1}
Stumpe,~M.~C., Smith,~J.~C., Van Cleve,~J.~E., \etal\ 2012,
\pasp, 124, 985

\bibitem[Vanderburg \& Johnson(2014a)]{vanderberg-a}
Vanderburg, A., \& Johnson, J.~A.\ 2014, \pasp, 126, 948

\bibitem[Vanderburg \etal(2014b)]{vanderburg-b}
Vanderburg, A., Montet, B.~T., Johnson, J.~A., \etal\ 2014, \arxiv{1412.5674}

\end{thebibliography}

\clearpage

% \begin{figure}[p]
% \begin{center}
% \includegraphics[width=\textwidth]{path/to/figure.pdf}
% \end{center}
% \caption{%
% A CAPTION.
% \figlabel{the-figure-label}}
% \end{figure}

\end{document}
